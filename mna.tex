\documentclass[12pt]{article}
\setcounter{secnumdepth}{0}
\usepackage{amsmath}
\usepackage{float}
\restylefloat{table}
\usepackage{mathtools}
\usepackage{amsfonts}
\usepackage{amssymb}
\usepackage{xcolor}
\usepackage{caption}
\usepackage[margin=1cm, font=small]{caption}
\usepackage{parskip}
\usepackage{amsmath}
\usepackage{hyperref}
\usepackage{fancyhdr}
\usepackage{sectsty}
\usepackage{caption}
\usepackage{soul}
\usepackage{nicematrix}
\usepackage{arydshln}
\usepackage[thinc]{esdiff}
\usepackage[spanish,es-tabla]{babel}
%\sectionfont{\fontsize{12}{15}\selectfont}
\linespread{1.25}
\newcommand{\R}{\mathbb{R}}
\newcommand{\C}{\mathbb{C}}
\newcommand{\by}{\times}
\newcommand{\talque}{\; \big/ \;}
\newcommand{\andd}{\; \wedge \;}
\newcommand{\talquep}{\; \big| \;}
\usepackage{geometry}
 \geometry{
 a4paper,
 %total
 left=25mm,
 top=28mm,
 right=25mm,
 bottom=30mm,
 }
 \newcommand{\sumn}[2]{\sum_{k=#1}^n #2}
 \newcommand{\F}{\mathscr{F}}

             
\usepackage{listings}
\newcommand*\lstinputpath[1]{\lstset{inputpath=#1}}

\lstinputpath{functions}
\usepackage{xcolor}
\usepackage{hyperref}
\hypersetup{
    colorlinks=true,
    linkcolor=black,
    urlcolor=black,
    citecolor=black,
}

\usepackage{titlesec}

\newcommand{\sectionbreak}{\clearpage}

\setcounter{secnumdepth}{4} % Establece el nivel de numeración de secciones
\setcounter{tocdepth}{4} % Establece el nivel de profundidad en el índice

\titleclass{\subsubsubsection}{straight}[\subsection] % Define la clase de la subsubsubsection
\newcounter{subsubsubsection}[subsubsection] % Define el contador para la subsubsubsection
\renewcommand{\thesubsubsubsection}{\thesubsubsection.\arabic{subsubsubsection}} % Define el formato de numeración de la subsubsubsection

\titleformat{\subsubsubsection}[runin]{\normalfont\normalsize\bfseries}{\thesubsubsubsection}{1em}{} % Define el formato de título de la subsubsubsection

\titlespacing*{\subsubsubsection}{0pt}{3.25ex plus 1ex minus .2ex}{1.5ex plus .2ex} % Define los espacios antes y después de la subsubsubsection
\usepackage{xcolor}

\definecolor{codegreen}{rgb}{0,0.6,0}
\definecolor{codegray}{rgb}{0.5,0.5,0.5}
\definecolor{codepurple}{rgb}{0.58,0,0.82}
\definecolor{backcolour}{rgb}{0.95,0.95,0.92}

\lstdefinestyle{mystyle}{
    backgroundcolor=\color{backcolour},   
    commentstyle=\color{codegreen},
    keywordstyle=\color{magenta},
    numberstyle=\tiny\color{codegray},
    stringstyle=\color{codepurple},
    basicstyle=\ttfamily\footnotesize,
    breakatwhitespace=false,         
    breaklines=true,                 
    captionpos=b,                    
    keepspaces=true,                 
    numbers=left,                    
    numbersep=5pt,                  
    showspaces=false,                
    showstringspaces=false,
    showtabs=false,                  
    tabsize=2
}

\lstset{style=mystyle}

\begin{document}
\begin{titlepage}
	\begin{center}
    \vspace*{0.8cm}
	{\Large Apuntes MNA \par}
  \end{center}

\end{titlepage}

\tableofcontents
\newpage

\section{Numeros complejos}
\[ i^{2n} = (-1)^n, \quad i^{2n+1} = (-1)^n i \]

\subsection{Propiedades}
\begin{itemize}
\item Sea $z = a_z + i b_z$, se cumple que:
\[z^{-1} = \frac{a_z}{a_z^2 + b_z^2} - i\frac{b_z}{a_z^2 + b_z^2}\]
\item \( \mathbb{C} \) es un cuerpo no ordenado.
\item El \textbf{conjugado} de \(z = a + bi\) es \(\overline{z} = a - bi\)
\end{itemize}
\subsection{Forma polar}
\(z = a + ib = \left(\underbrace{\sqrt{a^2+ b^2}}_{\rho}, \underbrace{arg(z)}_{\theta}\right) = (\rho, \theta)\) donde \(|arg(z)| = \arctg(b/a)\)
\subsubsection{Forma Polar Trigonométrica}
\[z = \rho \cos(\theta) + i \rho \sin(\theta)\]
\subsubsection{Forma Polar Exponencial}
\[z = \rho \; e^{i\theta}\]
\subsubsection{Propiedades}
\begin{itemize}
    \item \(\arg(z \; w) = \arg(z) + \arg(w)\)
    \item \(\arg(\frac{z}{w}) = \arg(z) - \arg(w)\)
    \item \(\arg(z^n) = n \; \arg(z)\)
    \item \(\arg(z^{-1}) = -\arg(z)\)
    \item \(z^n = |z|^n \; e^{i \; n \; \arg(z)}\) con \(n \in \mathbb{N}\)
    \item El \textbf{conjugado} de \(z = (\rho, \theta)\) es \(\overline{z} = (\rho, -\theta)\)
\end{itemize}
\subsubsection*{Propiedades del conjugado}
\begin{itemize}
    \item \(z = \overline{z} \iff Im(z) = 0\)
    \item \(\overline{z+w} = \overline{z} + \overline{w}\)
    \item \(\overline{z \; w} = \overline{z} \; \overline{w}\)
    \item \(\overline{\left(\frac{z}{w} \right)} = \frac{\overline{z}}{\overline{w}}\)
    \item \(z \; \overline{z} = |z|^2\)
    \item \( z + \overline{z} = 2 Re(z) \)
    \item \(z - \overline{z} = 2 \; i \; Im(z)\)
    \item \(|z| = |\overline{z}|\)
\end{itemize}
\subsubsection*{Radicación}
Sea \(w^n = z\) con \(n \in \mathbb{N}\), se deduce que:
\[w_k = \sqrt[n]{\rho} \; e^{i \left(\frac{\theta + 2k \pi}{n}\right)}, \quad k \in \{0, \dots, n-1\}\]
\subsubsection*{Logaritmación}
Sea \(e^w = z\), las soluciones son:
\[w_k = \ln|z| + i (\arg(z) + 2k\pi), \quad k \in \mathbb{Z}\]
A veces se trabaja en el valor principal del logaritmo dado por:
\[w_0 = \ln|z| + i \arg(z)\]
\subsubsection*{Potencia compleja}
\[z^w = e^{w \ln z}\]

\section*{Matrices}
\[ A = 
\begin{pmatrix}
    a_{11} & a_{12} & \dots & a_{1n} \\
    a_{21} & a_{22} & \dots & a_{2n} \\
    \vdots & \vdots & \vdots & \vdots \\
    a_{m1} & a_{m2} & \dots & a_{mn} \\
\end{pmatrix}, \quad a_{ij} \in K = \{\R,\C\}
\]
\begin{itemize}
    \item Notación: \(A = \left(a_{ij}\right)_{ij}\). Se interpreta como decir que el elemento dentro de los paréntesis \(a_{ij}\) se inserta en la posición \(ij\) de la matriz resultante.
    \item Vector fila: \(F_k = \left(a_{k1},a_{k2}, \dots, a_{kn}\right)\)
    \item Vector columna: \(C_k = \begin{pmatrix}
        a_{k1}\\a_{k2}\\\vdots\\a_{kn}
    \end{pmatrix}\)
\end{itemize}
\subsubsection*{Matriz transpuesta}
\[A^T = (a_{ji})_{ij}\]
Notación: \(A^T,A^t,A^*,A'\)
\[
    A = \begin{pmatrix}
        11 & -12 & 7 \\
        8 & 4 & -2
    \end{pmatrix} \implies A^T = \begin{pmatrix}
        11 & 8 \\
        -12 & 4\\
        7 & -2
    \end{pmatrix}\]
\textbf{Propiedades:}
\begin{itemize}
    \item \((A^T)^T = A\)
    \item \((A+B)^T = A^T + B^T\)
    \item \((\lambda A)^T =  \lambda \; A^T\)
    \item \((A \cdot B)^T = B^T \cdot A^T\)
\end{itemize}
 
\subsubsection*{Matriz Simétrica}
Solo definido para matrices cuadradas. \(A \in \R^{n \times n}\) es simétrica \(\iff A= A^T \iff a_{ij} = a_{ji} \; \forall i,j \leq n\)

\textbf{Observación:} Si \(A \in \R ^{m \times n}\), quedan definidas \(A \cdot A^T\) y \(A^T \cdot A\) que con \textbf{cuadradas} y \textbf{simétricas}.
\subsubsection*{Matriz Antisimétrica}
Definido sólo para matrices cuadradas. \(A \in \R^{n \times n}\)  es simétrica \(\iff a_{ij} = - a_{ji} \implies a_{kk} = 0\)
\subsubsection*{Traza}
\[Tr(A) = \sum_{i=1}^n a_{ii} = Suma \; diagonal\]
\textbf{Propiedades:}
\begin{itemize}
    \item \(Tr(A+B) = Tr(A) + Tr(B)\)
    \item  \(Tr(\lambda) = \lambda \; Tr(A)\)
    \item En general: \(Tr(A \cdot B) \neq Tr(A).Tr(B)\)
\end{itemize}
\subsubsection*{Matrices especiales}
Definido para matrices cuadradas únicamente:
\begin{itemize}
    \item Matriz triangular superior: \(\begin{pmatrix}
        a_{11} & a_{12} & a_{13}\\
        0 & a_{22} & a_{23} \\
        0 & 0 & a_{33}
    \end{pmatrix}\)
    \item Matriz triangular inferior: \(\begin{pmatrix}
        a_{11} & 0 & 0 \\
        a_{21} & a_{22} & 0 \\
        a_{31} & a_{32} & a_{33}
    \end{pmatrix}\)
    \item Matriz diagonal: \(\begin{pmatrix}
        a_{11} & 0 & 0 \\
        0 & a_{22} & 0 \\
        0 & 0 & a_{33}
    \end{pmatrix}\)
    \item Matriz escalar: es una matriz diagonal con todos los valores iguales.
\(\begin{pmatrix}
        a & 0 & 0 \\
        0 & a & 0 \\
        0 & 0 & a
    \end{pmatrix}\)
    \item Matriz identidad (\(I_n \in \R^{n \times n}\)):
    \(\begin{pmatrix}
        1 & 0 & 0 \\
        0 & 1 & 0 \\
        0 & 0 & 1
    \end{pmatrix}\)
\end{itemize}
\textbf{Propiedad:} Para toda \(A \in \R^{n \times n}, \quad A \cdot I_n = I_n \cdot A = A\)
\subsubsection*{Matriz Inversa}
Sea \(A \in K^{n \times n}\), si existe \(B \in K^{n \times n}\) tal que:
\[A \cdot B = B \cdot A = I \implies A \; \text{es inversible}\]
Notación: \(B = A^{-1}\). No siempre existe la inversa de una matriz.
\textbf{Propiedades:}
\begin{itemize}
    \item Es única
    \item \((A^{-1})^{-1} = A\)
    \item \((k.A) = \frac{1}{k} A^{-1}\)
    \item \((A \cdot B)^{-1} = B^{-1} \cdot A^{-1}\)
\end{itemize}
 
\subsubsection*{Definiciones}
\begin{itemize}
    \item Si existe \(A^{-1}\), se dice que \(A\) \textbf{es regular o inversible}.
    \item Si no existe \(A^{-1}\) se dice que \(A\) es \textbf{singular o no inversible}. 
    \item \(A\) es \textbf{ortogonal} \(\iff A^{-1} = A^T \iff A^T \cdot A = I\)  
\end{itemize}





 


\section*{Determinantes}
Notación: \(d(A) = det(A)\)
\textbf{Propiedades:} Sea \(A = \left(A_1, \dots, A_n\right) \in K^{n \times n}\):
\begin{itemize}
    \item \(d(I) = 1\)
    \item Si existe \(j \; \big| A_j = A_{j+1} \implies d(A) = 0\)
    \item Si \(A'_j = \alpha A_j \implies d(A_1, \dots,A'_j,\dots,A_n) = \alpha.d(A_1, \dots, A_n)\)
    
    Ejemplo:

\[det \left(
\begin{pmatrix}
    a_{11} & \colorbox{blue!20}{$2.a_{12}$} & a_{13} \\
    a_{21} & \colorbox{blue!20}{$2.a_{22}$} & a_{23} \\
    a_{31} & \colorbox{blue!20}{$2.a_{32}$} & a_{33} \\
\end{pmatrix} \right) = 2 . det\left(
\begin{pmatrix}
    a_{11} & \colorbox{blue!20}{$a_{12}$} & a_{13} \\
    a_{21} & \colorbox{blue!20}{$a_{22}$} & a_{23} \\
    a_{31} & \colorbox{blue!20}{$a_{32}$} & a_{33} \\
\end{pmatrix} \right)
\]
\item Si \(A_j = B_j + C_j \implies det\left(A_1, \dots , A_n\right) = det\left(A_1,\dots,B_j,\dots,A_n\right) + det\left(A_1,\dots,C_j,\dots,A_n\right) \)
Ejemplo:


\[det \left(
\begin{pmatrix}
    a_{11} & \colorbox{blue!20}{$b_{12} + c_{12}$} & a_{13} \\
    a_{21} & \colorbox{blue!20}{$b_{22} + c_{22}$} & a_{23} \\
    a_{31} & \colorbox{blue!20}{$b_{32} + c_{32}$} & a_{33} \\
\end{pmatrix} \right) =  det \left(
\begin{pmatrix}
    a_{11} & \colorbox{blue!20}{$b_{12}$} & a_{13} \\
    a_{21} & \colorbox{blue!20}{$b_{22}$} & a_{23} \\
    a_{31} & \colorbox{blue!20}{$b_{32}$} & a_{33} \\
\end{pmatrix} \right) + det\left(
\begin{pmatrix}
    a_{11} & \colorbox{blue!20}{$c_{12}$} & a_{13} \\
    a_{21} & \colorbox{blue!20}{$c_{22}$} & a_{23} \\
    a_{31} & \colorbox{blue!20}{$c_{32}$} & a_{33} \\
\end{pmatrix} \right)
\]
\end{itemize}
\subsubsection*{Cálculo de determinantes}
\[
A = \begin{pmatrix}
    a_{11} & a_{12} \\
    a_{21} & a_{22} \\
\end{pmatrix}
\]

El determinante de \(A\) se calcula como:
\[
\det(A) = a_{11} a_{22} - a_{12} a_{21}
\]

Para el caso de una matriz \(B \in K^{3 \times 3}\):
\[
B = \begin{pmatrix}
    a_{11} & a_{12} & a_{13} \\
    a_{21} & a_{22} & a_{23} \\
    a_{31} & a_{32} & a_{33} \\
\end{pmatrix}
\]

El cálculo de hace de la siguiente forma:

\[
\det(B) = a_{11} \cdot \det\begin{pmatrix} a_{22} & a_{23} \\ a_{32} & a_{33} \end{pmatrix} - a_{12} \cdot \det\begin{pmatrix} a_{21} & a_{23} \\ a_{31} & a_{33} \end{pmatrix} + a_{13} \cdot \det\begin{pmatrix} a_{21} & a_{22} \\ a_{31} & a_{32} \end{pmatrix}
\]

En general, el signo de cada factor es \((-1)^{i+j}\).
\subsubsection*{Propiedad}
\[A \; \text{es regular} \iff det(A) \neq 0\]


\section*{Espacios vectoriales}

\section*{Transformaciones lineales}
La idea es definir una función entre espacios vectoriales pero que perseven estructura.

Sea $ V, W $ espacios vectoriales. Una transformación lineal es una función $ T:V \to W $ que cumple:
\begin{enumerate}
    \item $ T(v_1 + v_2) = Tv_1 + Tv_2 $
    \item $ T(\lambda v) = \lambda . T(v) $
\end{enumerate}

\subsection*{Núcleo de una TL}
Se define como \textbf{núcleo de una TL $ T:V \to W $} al conjunto:
\begin{math}
    N(T) = \left\{ v \in V \talque T(v) = 0_{W} \right\}
\end{math}
\textbf{Proposición:} $ N(T) $ es un subespacio vectorial.
\subsection*{Imagen de una TL}
Se define como \textbf{Imagen de T} a:
\begin{math}
    R(T) = \left\{ T(v) \talque v \in V \right\}
\end{math}
\subsection{Clasificación de las Transformaciones Lineales}
\begin{itemize}
    \item $ T:V \to W $ es inyectiva si dados $ v_1,v_2 \in V : T(v_1) =T(v_2) \implies v_1 = v_2 $

    \textbf{Proposición:} $ T \; \text{es inyectiva} \iff N(T) = \left\{ 0 \right\}$ 
\item $ T: V \to W $ es \textbf{sobreyectiva} si $ R(T) = W $

    \textbf{Recordar:} $ S \subseteq T \andd dim(S) = dim(T) \implies S = T $

\item $ T $ es \textbf{biyectiva (isomorfismo)} si es \underline{inyectiva y sobreyestiva}.

\end{itemize}

\subsection{Teorema de las dimensiones}
Sea $ V $ espacio vectorial de dimensión finita y $ T : V \to W $ una TL con $ R(T) $ de dimensión finita:

\[
    dim(V) = dim(N(T)) + dim(R(T))
\]

\subsection{Teorema}
Si $ T: V \to W $ y $ dim(V) = dim(W) $, entonces:
\[
    T \text{ es Monoformismo} \iff T \text{ es Epimorfismo} \iff T \text{ es Isomorfismo}
\]

\subsection{Teorema fundamental de las TL}
Sea $ V,W $ dos espacios vectoriales. Sea $ B =\left\{ v_1, \dots, v_n \right\} $ una base de $ V $ y $ A = \left\{ w_1, \dots, w_n \right\} \subseteq W$. Entonces existe una única transformación lineal $ T:V \to W $ que cumple:
\[
    T(v_1) = w_1, \; \dots, \; T(v_n) = w_n
\]
\subsection{Matriz Asociada a una Transformación Lineal}
Sea $ T :V \to W $ una TL, existe un $ \tilde{T}: \R^n \to \R^m $ con $ n = dim(V) $ y $ m = dim(W) $

Consideremos $ B_V = \left\{ v_1,\dots,v_n \right\}, $$ B_W = \left\{ w_1, \dots, w_n \right\}$ bases de $ V $ y $ W $ respectivamente:

\begin{itemize}
    \item Dado $ v \in V $, existe:
        \[
            (\alpha_1, \dots, \alpha_n) \talque v = \alpha_1 v_1 + \dots + \alpha_n v_n
    \]
\item Dado $ w \in W $, existe:

        \[ 
            (\beta_1, \dots, \beta_n) \talque w = \beta_1 w_1 + \dots + \beta_n w_n
    \]

\item Además:
    \[
        T(v_i) = a_{i1}w_1 + \dots + a_{in}w_n
    \]
 
\end{itemize}

Es decir:

\[
    \begin{pmatrix} \beta_1 \\ \vdots \\ \beta_m \end{pmatrix} =
    \underbrace{\begin{pmatrix} a_{11} & \dots & a_{1n} \\
            \vdots & \vdots & \vdots \\
            a_{m1} & \dots & a_{mn}
        \end{pmatrix}}_{A} \cdot \begin{pmatrix} 
    \alpha_1 \\ \vdots \\ \alpha_m \end{pmatrix} \text{ que equivale a decir: } \left[T(v)\right]_{B_W} = A \cdot \left[v\right]_{B_V}\]


A la matriz $ A $ se la llama \textbf{Matriz asociada a la Transformación en las bases $ B_V $ y $ B_W $}

Notación: $ A = M(T)_{B_VB_W} $

Informalmente:

\[
    \text{''Todas las TL entre espacios de dim. finita son de la forma $T(X) = A \cdot X$''}
\]

Sea $ T :V \to W $, con $ B_V = \left\{ v_1, \dots, v_n \right\}$ y  $ B_W = \left\{ w_1, \dots, w_n \right\}$ notar:

\[
    M(T)_{B_VB_W} = \begin{pmatrix} \left[T(v_1)\right]_{B_W} & \dots & \left[T(v_n)\right]_{B_W} \end{pmatrix}
\]

\section*{Autovalores y Autovectores}
\subsection{Diagonalización}    
Una \textbf{matriz diagonal} (solo aplica a matrices cuaradas) tiene la siguiente forma:
\[
    D = \begin{pmatrix} a_{11} & 0 & 0\\
                        0 & a_{22} & 0\\
                        0 & 0 & a_{33}\end{pmatrix}
\]
\subsubsection{Ventajas de matrices diagonales}
\begin{enumerate}
    \item Sistemas de ecuaciones:
        \[
     \begin{pmatrix} a_{11} & 0 & 0\\
                        0 & a_{22} & 0\\
                        0 & 0 & a_{33}\end{pmatrix} \cdot \begin{pmatrix}  x \\ y \\ z \end{pmatrix} = 
                        \begin{pmatrix}  b_1 \\ b_2 \\ b_3 \end{pmatrix} \implies b_1 = a_{11}x, \; b_2 = a_{22}y, \; b_3 = a_{33}z
        \]
    \item Cálculo Funcional:
        \[
            f: \R \to \R, \; D \text{ diagonal} \implies f(D) = \begin{pmatrix} a_{11} & 0 & 0\\
                        0 & a_{22} & 0\\
                        0 & 0 & a_{33}\end{pmatrix}
        \]

        Ejemplo:
        \[
            D^2  =     \begin{pmatrix} a_{11}^2 & 0 & 0\\
                        0 & a_{22}^2 & 0\\
                        0 & 0 & a_{33}^2\end{pmatrix}
        \]

\end{enumerate}

\subsection{Matrices Semejantes}
Sean $ A,B\ in K^{n \by n} $, se dice que \textbf{$A$ es semejante a $B$} si ocurre que:
\[
    \exists  P \in K^{n \by n}  \talque B = P^{-1} \cdot A \cdot P \iff A = P \cdot B \cdot P^{-1}.  
\]
A $ P $ se le llama \textbf{Matriz Cambio de Base}.
\subsection{Autovalores y Autovectores}
Sea $ A \in K^{n \by n} $. Se dice que $ X \in K^{n} - \left\{ 0 \right\} $ es \textbf{autovector de $A$}, si existe $ \lambda \in K $ tal que:
\[
    A \cdot X = \lambda \cdot X
\]
Al escalar $\lambda$ se lo llama \textbf{autovalor asociado al autovector $X$}.
\[
\text{A } \left<A \cdot X\right> = \left<X\right> \text{ se los llama \textbf{subespacios invariantes.}}\]

\end{document}




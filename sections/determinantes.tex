\section*{Determinantes}
Notación: \(d(A) = det(A)\)
\textbf{Propiedades:} Sea \(A = \left(A_1, \dots, A_n\right) \in K^{n \times n}\):
\begin{itemize}
    \item \(d(I) = 1\)
    \item Si existe \(j \; \big| A_j = A_{j+1} \implies d(A) = 0\)
    \item Si \(A'_j = \alpha A_j \implies d(A_1, \dots,A'_j,\dots,A_n) = \alpha.d(A_1, \dots, A_n)\)
    
    Ejemplo:

\[det \left(
\begin{pmatrix}
    a_{11} & \colorbox{blue!20}{$2.a_{12}$} & a_{13} \\
    a_{21} & \colorbox{blue!20}{$2.a_{22}$} & a_{23} \\
    a_{31} & \colorbox{blue!20}{$2.a_{32}$} & a_{33} \\
\end{pmatrix} \right) = 2 . det\left(
\begin{pmatrix}
    a_{11} & \colorbox{blue!20}{$a_{12}$} & a_{13} \\
    a_{21} & \colorbox{blue!20}{$a_{22}$} & a_{23} \\
    a_{31} & \colorbox{blue!20}{$a_{32}$} & a_{33} \\
\end{pmatrix} \right)
\]
\item Si \(A_j = B_j + C_j \implies det\left(A_1, \dots , A_n\right) = det\left(A_1,\dots,B_j,\dots,A_n\right) + det\left(A_1,\dots,C_j,\dots,A_n\right) \)
Ejemplo:


\[det \left(
\begin{pmatrix}
    a_{11} & \colorbox{blue!20}{$b_{12} + c_{12}$} & a_{13} \\
    a_{21} & \colorbox{blue!20}{$b_{22} + c_{22}$} & a_{23} \\
    a_{31} & \colorbox{blue!20}{$b_{32} + c_{32}$} & a_{33} \\
\end{pmatrix} \right) =  det \left(
\begin{pmatrix}
    a_{11} & \colorbox{blue!20}{$b_{12}$} & a_{13} \\
    a_{21} & \colorbox{blue!20}{$b_{22}$} & a_{23} \\
    a_{31} & \colorbox{blue!20}{$b_{32}$} & a_{33} \\
\end{pmatrix} \right) + det\left(
\begin{pmatrix}
    a_{11} & \colorbox{blue!20}{$c_{12}$} & a_{13} \\
    a_{21} & \colorbox{blue!20}{$c_{22}$} & a_{23} \\
    a_{31} & \colorbox{blue!20}{$c_{32}$} & a_{33} \\
\end{pmatrix} \right)
\]
\end{itemize}
\subsubsection*{Cálculo de determinantes}
\[
A = \begin{pmatrix}
    a_{11} & a_{12} \\
    a_{21} & a_{22} \\
\end{pmatrix}
\]

El determinante de \(A\) se calcula como:
\[
\det(A) = a_{11} a_{22} - a_{12} a_{21}
\]

Para el caso de una matriz \(B \in K^{3 \times 3}\):
\[
B = \begin{pmatrix}
    a_{11} & a_{12} & a_{13} \\
    a_{21} & a_{22} & a_{23} \\
    a_{31} & a_{32} & a_{33} \\
\end{pmatrix}
\]

El cálculo de hace de la siguiente forma:

\[
\det(B) = a_{11} \cdot \det\begin{pmatrix} a_{22} & a_{23} \\ a_{32} & a_{33} \end{pmatrix} - a_{12} \cdot \det\begin{pmatrix} a_{21} & a_{23} \\ a_{31} & a_{33} \end{pmatrix} + a_{13} \cdot \det\begin{pmatrix} a_{21} & a_{22} \\ a_{31} & a_{32} \end{pmatrix}
\]

En general, el signo de cada factor es \((-1)^{i+j}\).
\subsubsection*{Propiedad}
\[A \; \text{es regular} \iff det(A) \neq 0\]


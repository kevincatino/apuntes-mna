\section*{Espacios vectoriales}
Sea un cuerpo \(K \in \{\R,\C\}\), un espacio vectorial es una cuaterna de la forma \(\left(V,\oplus,K,\otimes\right)\) donde:
\begin{itemize}
    \item \(\oplus : V \by V \to V\)
    \begin{itemize}
        \item \(u + (v + w) = (u+v) + w\)
        \item Existe elemento neutro \(n \in V \; \big| \; n +v = v+n = v\)
        \item Existe elemento opuesto \(-v \forall v \in V \; \big| \; v + (-v) = n\)
    \end{itemize}
    
    \item \(\otimes: K \by V \to V\)
    \begin{itemize}
        \item \(\lambda(u+v) = \lambda u + \lambda v\)
        \item \(\lambda(\beta v) = (\lambda \beta) v\)
        \item Sea \(1 \in K \; \text{el neutro multiplicativo} \implies 1.v = v\)

        \item Sea \(0 \in K \; \text{el elemento nulo} \implies 0.v = \mathbf{0} \in V\)
    \end{itemize}
    
\end{itemize}
\subsection*{Subespacios}
Sea \(V\) espacio vectorial y \(S \subseteq V\). Decimos que \(S\) es un \textbf{subespacio vectorial} si:
\begin{enumerate}
    \item \(\forall u,v \in S : (u+v) \in S\)
    \item \(\forall v \in S, \lambda \in K: (\lambda . v) \in S\)
\end{enumerate}
\subsubsection*{Combinación lineal}
Sea \((V,\oplus,K,\otimes)\) un espacio vectorial, y sea \(A = \{v_1,\dots,v_n\} \subseteq V\). Decimos que \(v\in V\) es una \textbf{combinación lineal de A} si existen \(\alpha_1,\dots,\alpha_n \in K\) tales que:
\[v = \alpha_1.v_1 + \dots + \alpha_n . v_n\] 
\begin{itemize}
    \item \(A\) es \textbf{linealmente dependiente (LD)} si:
     \[\exists \alpha_i \neq 0 \talque \alpha_1.v_1 + \dots + \alpha_n . v_n = 0\]
    \item \(A\) es \textbf{linealmente independiente (LI)} si:
     \[\alpha_1.v_1 + \dots + \alpha_n v_n = 0 \iff \alpha_1 = \dots = \alpha_n = 0\] 
\end{itemize}
\subsubsection*{Subespacio generado}
Se define como \textbf{subespacio generado por \(A\)} al conjunto de todas las combinaciones lineales de A.

\[A = \left\{\sum_{i=1}^n \alpha_i . v_i \talque \alpha_i \in K\right\}\]

Notación: \(\left<A\right>, gen(A)\)
\subsubsection*{Base y Dimensión}
Sea \((V,\oplus,K,\otimes)\) un espacio vectorial y \(B = \left\{v_1,\dots,v_n\right\} \subseteq V\). Decimos que \(B\) es una \textbf{base de \(V\)} si:
\begin{enumerate}
    \item Es \textbf{linealmente independiente (LI)}.
    \item Genera \(V\), es decir: \(gen(B) = V\) 
\end{enumerate} 
\textbf{Definición:} Se define como \textbf{dimensión de \(V\) un espacio vectorial} al cardinal de una base de \(V\). Notación: \(dim(V)\)

\subsubsection*{Proposición}
\[\text{Todo espacio vectorial tiene infinitas bases pero todas tienen el mismo cardinal.}\]
\subsubsection*{Propiedades}
Sean \(S,T \subseteq V\) subsespacios entonces:
\begin{itemize}
    \item \(S \subseteq T \implies dim(S) \leq dim(T)\)
    \item \(S \subseteq T \; \wedge \; dim(S) = dim(T) \implies S = T\)
\end{itemize}
\subsubsection*{Coordenadas de un bector en una Base}
Sea \((V,\oplus, K,\otimes)\) un espacio vectorial, \(B = \left\{v_1,\dots,v_n\right\}\) una base de \(V\). Sea \(v \in V\), sabemos que existen \(\alpha_1, \dots, \alpha_n\) tal que:
\[v = \alpha_1.v_1 + \dots + \alpha_n.v_n\]

A los escalares \(\alpha_1, \dots, \alpha_n\) se los llama \textbf{coordenadas de \(V\) en la base \(B\)} y se suele usar el vector \(\alpha = (\alpha_1, \dots, \alpha_n)\)

Notación: \([v]_B = (\alpha_1,\dots,\alpha_n)\)


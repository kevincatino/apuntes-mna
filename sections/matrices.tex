
\section*{Matrices}
\[ A = 
\begin{pmatrix}
    a_{11} & a_{12} & \dots & a_{1n} \\
    a_{21} & a_{22} & \dots & a_{2n} \\
    \vdots & \vdots & \vdots & \vdots \\
    a_{m1} & a_{m2} & \dots & a_{mn} \\
\end{pmatrix}, \quad a_{ij} \in K = \{\R,\C\}
\]
\begin{itemize}
    \item Notación: \(A = \left(a_{ij}\right)_{ij}\). Se interpreta como decir que el elemento dentro de los paréntesis \(a_{ij}\) se inserta en la posición \(ij\) de la matriz resultante.
    \item Vector fila: \(F_k = \left(a_{k1},a_{k2}, \dots, a_{kn}\right)\)
    \item Vector columna: \(C_k = \begin{pmatrix}
        a_{k1}\\a_{k2}\\\vdots\\a_{kn}
    \end{pmatrix}\)
\end{itemize}
\subsubsection*{Matriz transpuesta}
\[A^T = (a_{ji})_{ij}\]
Notación: \(A^T,A^t,A^*,A'\)
\[
    A = \begin{pmatrix}
        11 & -12 & 7 \\
        8 & 4 & -2
    \end{pmatrix} \implies A^T = \begin{pmatrix}
        11 & 8 \\
        -12 & 4\\
        7 & -2
    \end{pmatrix}\]
\textbf{Propiedades:}
\begin{itemize}
    \item \((A^T)^T = A\)
    \item \((A+B)^T = A^T + B^T\)
    \item \((\lambda A)^T =  \lambda \; A^T\)
    \item \((A \cdot B)^T = B^T \cdot A^T\)
\end{itemize}
 
\subsubsection*{Matriz Simétrica}
Solo definido para matrices cuadradas. \(A \in \R^{n \times n}\) es simétrica \(\iff A= A^T \iff a_{ij} = a_{ji} \; \forall i,j \leq n\)

\textbf{Observación:} Si \(A \in \R ^{m \times n}\), quedan definidas \(A \cdot A^T\) y \(A^T \cdot A\) que con \textbf{cuadradas} y \textbf{simétricas}.
\subsubsection*{Matriz Antisimétrica}
Definido sólo para matrices cuadradas. \(A \in \R^{n \times n}\)  es simétrica \(\iff a_{ij} = - a_{ji} \implies a_{kk} = 0\)
\subsubsection*{Traza}
\[Tr(A) = \sum_{i=1}^n a_{ii} = Suma \; diagonal\]
\textbf{Propiedades:}
\begin{itemize}
    \item \(Tr(A+B) = Tr(A) + Tr(B)\)
    \item  \(Tr(\lambda) = \lambda \; Tr(A)\)
    \item En general: \(Tr(A \cdot B) \neq Tr(A).Tr(B)\)
\end{itemize}
\subsubsection*{Matrices especiales}
Definido para matrices cuadradas únicamente:
\begin{itemize}
    \item Matriz triangular superior: \(\begin{pmatrix}
        a_{11} & a_{12} & a_{13}\\
        0 & a_{22} & a_{23} \\
        0 & 0 & a_{33}
    \end{pmatrix}\)
    \item Matriz triangular inferior: \(\begin{pmatrix}
        a_{11} & 0 & 0 \\
        a_{21} & a_{22} & 0 \\
        a_{31} & a_{32} & a_{33}
    \end{pmatrix}\)
    \item Matriz diagonal: \(\begin{pmatrix}
        a_{11} & 0 & 0 \\
        0 & a_{22} & 0 \\
        0 & 0 & a_{33}
    \end{pmatrix}\)
    \item Matriz escalar: es una matriz diagonal con todos los valores iguales.
\(\begin{pmatrix}
        a & 0 & 0 \\
        0 & a & 0 \\
        0 & 0 & a
    \end{pmatrix}\)
    \item Matriz identidad (\(I_n \in \R^{n \times n}\)):
    \(\begin{pmatrix}
        1 & 0 & 0 \\
        0 & 1 & 0 \\
        0 & 0 & 1
    \end{pmatrix}\)
\end{itemize}
\textbf{Propiedad:} Para toda \(A \in \R^{n \times n}, \quad A \cdot I_n = I_n \cdot A = A\)
\subsubsection*{Matriz Inversa}
Sea \(A \in K^{n \times n}\), si existe \(B \in K^{n \times n}\) tal que:
\[A \cdot B = B \cdot A = I \implies A \; \text{es inversible}\]
Notación: \(B = A^{-1}\). No siempre existe la inversa de una matriz.
\textbf{Propiedades:}
\begin{itemize}
    \item Es única
    \item \((A^{-1})^{-1} = A\)
    \item \((k.A) = \frac{1}{k} A^{-1}\)
    \item \((A \cdot B)^{-1} = B^{-1} \cdot A^{-1}\)
\end{itemize}
 
\subsubsection*{Definiciones}
\begin{itemize}
    \item Si existe \(A^{-1}\), se dice que \(A\) \textbf{es regular o inversible}.
    \item Si no existe \(A^{-1}\) se dice que \(A\) es \textbf{singular o no inversible}. 
    \item \(A\) es \textbf{ortogonal} \(\iff A^{-1} = A^T \iff A^T \cdot A = I\)  
\end{itemize}





 

